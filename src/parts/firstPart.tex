\section{Une section}

% Use H to place the figure HERE
\begin{figure}[H]
    \centering
    \includegraphics[width=\textwidth]{latexsup.png}
    \caption{Quod Erat Demonstrandum}
\end{figure}

% to prevent the picture to go at the end of this part
\FloatBarrier

\clearpage

\subsection{Une sous-section}

Une liste :
\begin{itemize} % liste sans ordre
    \item Niveau 1 - \href{https://fr.wikipedia.org/wiki/USB}{USB}
    \begin{itemize}
        \item Niveau 2 - \href{https://fr.wikipedia.org/wiki/Ethernet}{Ethernet}
        \begin{itemize}
            \item Un élément de niveau 3 - IP
            \begin{enumerate} % Liste ordonnée
                \item Un élément de niveau 4 - \href{https://en.wikipedia.org/wiki/TCP}{TCP}
                \item Un second élément de niveau 4 - \href{https://en.wikipedia.org/wiki/UDP}{UDP}
            \end{enumerate}
        \end{itemize}
        \item Retour au niveau deux - \href{https://fr.wikipedia.org/wiki/Spanning_Tree_Protocol}{STP}
    \end{itemize}
    \item[(NomChoisi)] Un autre élément de niveau 1 - \href{https://fr.wikipedia.org/wiki/Carrier_Sense_Multiple_Access_with_Collision_Avoidance}{CSMA/CA}
\end{itemize}

Une liste sur deux colonnes :
\begin{itemize}[twocol]
    \item Je suis sur la première colonne !
    \item Moi aussi.
    \item Et moi ?
    \item Je ne sais pas moi.
    \item Toi tu es de l'autre côté, sur la première colonne !
    \item Moi, je suis avec toi sur la deuxième colonne.
\end{itemize}


\subsection{Une autre sous-section}
Si votre rapport est confidentiel, vous pouvez cachez les éléments importants comme ce qui va suivre.
La recette du bonheur c'est \censor{un peu de sel}, de l'amour et un peu \censor{d'amitié}.

Le missile \censor{Liberty}, avec un diamètre de charge (CD) de
\censor{80}~mm, a révélé une capacité de pénétration dans le blindage
en acier.
\begin{table}[ht]
    \begin{center}
        \textbf{Table 1. \censor{Liberty} Missile Test Data}\\
        \censorbox{%
            \small\begin{tabular}{cc}
                Standoff & Penetration \\
                (CD) & (CD) \\
                \hline
                5.0 & 1.30 \\
                6.0 & 1.19 \\
                6.8 & 1.37\\
            \end{tabular}%
        }
        \end{center}
\end{table}


\subsubsection{Une sous-sous-section}
Un excellent professeur proclama un jour:
\begin{center}
Il fait trop chaud pour faire du réseau.
\end{center}

A l'\textbf{extrême gauche} on a:
\begin{flushleft}
    Coucou comment ça va ?
\end{flushleft}

Tandis qu'à l'\underline{extrême droite} on a le \href{https://rassemblementnational.fr/}{\footnote{Rassemblement National}{RN}} et aussi cette mise en forme:

\begin{flushright}
    Vous ne trouvez pas que petit, on a tous voulu changer la société avant que ce soit elle qui nous change ?
\end{flushright}

\subsubsection{Une autre sous-sous-section}
\paragraph{Un paragraphe}
Une citation c'est bien, mais bien citer c'est mieux :

\begin{quoting}
    Mais, vous savez, moi je ne crois pas qu’il y ait de bonne ou de mauvaise situation. Moi, si je devais résumer ma vie aujourd’hui avec vous, je dirais que c’est d’abord des rencontres, des gens qui m’ont tendu la main, peut-être à un moment où je ne pouvais pas, où j’étais seul chez moi. Et c’est assez curieux de se dire que les hasards, les rencontres forgent une destinée… Parce que quand on a le goût de la chose, quand on a le goût de la chose bien faite, le beau geste, parfois on ne trouve pas l’interlocuteur en face, je dirais, le miroir qui vous aide à avancer. Alors ce n’est pas mon cas, comme je le disais là, puisque moi au contraire, j’ai pu ; et je dis merci à la vie, je lui dis merci, je chante la vie, je danse la vie… Je ne suis qu’amour ! Et finalement, quand beaucoup de gens aujourd’hui me disent : « Mais comment fais-tu pour avoir cette humanité ? » Eh bien je leur réponds très simplement, je leur dis que c’est ce goût de l’amour, ce goût donc qui m’a poussé aujourd’hui à entreprendre une construction mécanique, mais demain, qui sait, peut-être simplement à me mettre au service de la communauté, à faire le don, le don de soi…
    \begin{flushright}
        --- Otis, Astérix Mission Cléopatre
    \end{flushright}
\end{quoting}

Si vous appréciez la façon "Markdown" de présenter les citations, je vous propose la même chose ici :
\begin{quoted}
    Ceci est une citation comme usuellement vue sur \texttt{Notion} ou en \texttt{Markdown}.
\end{quoted}

\subparagraph{Un sous-paragraphe}

\begin{dialogue}
    \speak{Un Allemand} \direct{s'esclaffe} Tous les allemands ne sont pas Nazis !
    \speak{Hubert Bonisseur de La Bath} Oui, je connais cette théorie
\end{dialogue}

\section*{Une section non numérotée}
\addcontentsline{toc}{section}{Une section non numérotée} % Si on veut qu'elle apparaisse dans le sommaire.
On peut créer une mise en forme attirant l'attention sur un point important à expliquer :

\begin{callout}{Contrôle de flux $\neq$ contrôle de congestion}
    \begin{itemize}
        \item Le \textbf{contrôle de flux} signifie essentiellement que TCP s'assure qu'un expéditeur ne submerge pas un destinataire en envoyant des paquets plus vite qu'il ne peut les consommer. Il concerne le nœud final.
        \item Le \textbf{contrôle de congestion} vise à empêcher un nœud de submerger le réseau (c'est-à-dire les liens entre deux nœuds).
    \end{itemize}
\end{callout}

Ou plus sobrement :

\begin{myboxedtext}
    Avoir un joli rapport $\Rightarrow$ $+50$ points de charisme.
\end{myboxedtext}